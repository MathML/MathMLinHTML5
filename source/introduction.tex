\section{Introduction}

\subsection{Rationale for this Technical Document}

{\em This section is non-normative.}

The Mathematical Markup Language is the most popular XML format for describing
mathematical notation and capturing both its structure and content
\cite{MathML3}. It has been integrated in various other standards such as
\cite{HTML5}, \cite{EPUB3}, \cite{ODF1} or \cite{Daisy3}. Although
the stated goal is ``to enable mathematics to be served, received, and
processed on the World Wide Web'', the MathML specification has
two serious shortcomings that make it hard to implement presentation MathML in
web rendering engines:

\begin{enumerate}
\item {\em The MathML specification intentionally does not contain any detailed
  rendering rules}. As a consequence, the fact that web rendering engines are
  compliant with the MathML specification does not necessarily mean that they
  will have the rendering quality expected by most readers.
  For example, the specification essentially just says that
  ``mfrac element is used for fractions'' and that the default medium
  linethickness ``is left up to the rendering agent'' \cite{MathML3}.
  As a comparison, to determine the exact spacing and thickness of
  fractions and stacks the
  \TeX Book's Appendix G \cite{TeXBook} relies
  on parameters $\sigma_8, \sigma_9, \xi_8, {3\xi_8}, {7\xi_8}, \sigma_{11}$ and
  $\sigma_{12}$ while the MATH table of the Open Font Format
  \cite{OpenFontFormat3} extends these to parameters
  {\tt FractionNumeratorDisplayStyleShiftUp\lxAddClass{MATH}},
  {\tt FractionNumeratorShiftUp\lxAddClass{MATH}},
  {\tt FractionNumeratorDisplayStyleGapMin\lxAddClass{MATH}},
  {\tt FractionNumeratorGapMin\lxAddClass{MATH}},
  {\tt FractionRuleThickness\lxAddClass{MATH}},
  {\tt FractionDenominatorDisplayStyleGapMin\lxAddClass{MATH}},
  {\tt FractionDenominatorGapMin\lxAddClass{MATH}},
  {\tt FractionDenominatorDisplayStyleShiftDown\lxAddClass{MATH}},
  {\tt FractionDenominatorShiftDown\lxAddClass{MATH}},
  {\tt StackTopDisplayStyleShiftUp\lxAddClass{MATH}},
  {\tt StackTopShiftUp\lxAddClass{MATH}},
  {\tt StackDisplayStyleGapMin\lxAddClass{MATH}},
  {\tt StackGapMin\lxAddClass{MATH}},
  {\tt StackBottomDisplayStyleShiftDown\lxAddClass{MATH}} and
  {\tt StackBottomShiftDown\lxAddClass{MATH}}.
\item {\em The MathML specification is designed as an independent
  XML language and browser vendors have almost not been involved in the
  standardization process},
  except for integration in \cite{HTML5}.
  Instead, the specification is sometimes biased by MathML rendering and
  authoring tools behaving quite differently from web rendering engines.
  Hence it is not always obvious whether all features are fundamental or
  whether they fit well into the web rendering engine codebase.
  For example, the {\tt <mfenced>} element is just a
  ``convenient form in which to express common constructs involving fences''
  but is strictly equivalent to an expanded form with {\tt <mrow>} and
  {\tt <mo>} elements. It requires web rendering engines
  to create many ``anonymous''
  rendering frames and keep them up-to-date, to duplicate the logic
  for drawing and exposing the content of fenced expressions etc.
  Another example is the first cell in a
  {\tt <mlabeledtr>} row which is not necessarily positioned the same way as
  in HTML table and actually is supposed not to be involved in metric
  computation or border drawing of the table \cite{MathML3}.
\end{enumerate}

This ``MathML in HTML5'' implementation note intends to address these issues by
being as accurate as
possible on the visual rendering of mathematical formulas using additional
rules from the \TeX Book's Appendix G \cite{TeXBook} and from
the Open Font Format \cite{OpenFontFormat3}.
Focus has been put on keeping compatible with existing technologies of web
rendering engines \cite{HTML5} by relying as much as possible on CSS, text \&
table layout and box models. As a consequence, parts of presentation MathML
that do not fit well in this framework or are rarely used in practice
have been ommited ; details on these and suggestions for standardization bodies
are provided in the Appendix.

Future versions of this document may describe support for a larger subset
of presentation MathML by including features that are important in a web context
e.g. CSS-compatible line breaking.

\subsection{Remarks}

{\em This section is non-normative.}

This document focuses on the most important points to implement
MathML in web rendering engines. Hence it is intentionally short and readers are
invited to check the MathML 3 specification for details \cite{MathML3}.
As a convenience, quotations from the MathML 3 specification are referred to
with the following style:
\begin{quote}
  This is a quotation from the MathML 3 specification.
\end{quote}

This document relies on many definitions taken from the MATH specification
of the Open Font Format \cite{OpenFontFormat3}. These values are written with
the following style:
{\tt This is a definition from the MATH specification\lxAddClass{MATH}}.

This document only deals with the visual rendering of presentation MathML
and briefly mentions interactions (javascript, links etc). It does not specify
how presentation MathML must be exposed to assitive technologies nor does it
explain how the semantic provided by content MathML can be used.

This document has been generated by
\href{http://dlmf.nist.gov/LaTeXML/}{LaTeXML} from
\href{https://github.com/MathML/MathMLinHTML5}{LaTeX sources}. It
should be read in a browser that supports web technologies
such as HTML, CSS, SVG and MathML.

\subsection{Terminology}

The key words MUST, MUST NOT, REQUIRED, SHALL, SHALL NOT, SHOULD, SHOULD NOT,
RECOMMENDED, MAY, and OPTIONAL, are to be interpreted as described in
\cite{IETF RFC 2119}.
