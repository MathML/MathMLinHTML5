\section{Relation with other technologies}

\subsection{HTML5 Tree}

\subsubsection{DOM, HTML, SVG and Javascript}

User agents must use a HTML5 \cite{HTML5} parser to build a DOM tree
\cite{DOM1} from the source code of web pages. In particular, they must follow
the rules describing when elements must be in the MathML namespace
{\tt http://www.w3.org/1998/Math/MathML} and must
recognize the entity definitions from the HTML MathML entity set of
\cite{XMLEntities}. They must also take into account the following integration
points between SVG, MathML and HTML as allowed by \cite{ValidatorSchemas}:

\begin{enumerate}
\item The {\tt <math>} element can be used at any position permitted for
  phrasing content or inside SVG {\tt <foreignObject>} elements.
\item The {\tt <svg>} element can be used inside {\tt <annotation-xml>}
  elements with encoding {\tt SVG1.1} or {\tt image/svg+xml}.
\item The {\tt <html>} element and flow content can be used inside
  {\tt <annotation-xml>} elements with encoding {\tt application/xhtml+xml}
  or {\tt text/html}.
\item Any phrasing element can be used inside {\tt <mtext>} elements.
\end{enumerate}

From this DOM tree, user agents must provide a visual representation of the
document. The DOM tree may be dynamically modified using Javascript
\cite{ECMA262} and the user agents must keep the visual representation in
synchronization with the DOM tree.

When evaluating the SVG {\tt requiredExtensions} attribute \cite{SVG11},
user agents must claim support for the extension of name
{\tt http://www.w3.org/1998/Math/MathML}.
An algorithm to decide the visible child of the {\tt <semantics>} element is
proposed in section \ref{semantics}.

\subsubsection{MathML}

All MathML elements accept the {\tt id}, {\tt class} and {\tt style} attributes
\cite{MathML3}. They must be interpreted as described in section 3.2.5 of the
HTML5 specification \cite{HTML5} and in particular they specify a unique
identifier (to identify elements in links and scripting), affect
CSS selectors, affect {\tt getElementsByClassName()} and enable authors to do
inline styling.

MathML 3 allows to use the {\tt href} attribute on any MathML element
\cite{MathML3}. In the present document, it is only required to implement
{\tt href} on the {\tt mrow} element with the behavior described in 4.8
of the HTML5 specification for the {\tt a} element \cite{HTML5}. It is
recommended to make links visually distinguisable by default, for example by
adding a rule in the user agent stylesheet (section \ref{UAStylesheet}) such as
\begin{lstlisting}
mrow[href] {
  color: blue;
}
\end{lstlisting}

The toplevel {\tt math} element accepts the {\tt altimg}, {\tt altimg-width},
{\tt altimg-height}, {\tt altimg-valign} and {\tt alttext} attributes.
These attributes allow to specify fallback content and must be ignored by
user agents for rendering purpose. User agents that do not follow the current
implementation note, may follow implementation suggestions from Appendix
\ref{ImageFallback} in order to use that fallback image.

The toplevel {\tt math} element also accepts the {\tt display} attribute,
{\tt mathcolor}, {\tt mathbackground} attributes as well as other attributes
from the {\tt mstyle} element. These attributes must be supported and this
may be achieved using specific rules in the user agent stylesheet
as described in section \ref{UAStylesheet}.

In general MathML elements or attributes that are not mentioned in this
document may just be ignored. This includes deprecated attributes or
Content Markup described in chapter 4 of the MathML 3 specification
\cite{MathML3}.

User agents must contain an operator dictionary describing the default
properties of operators. For interoperability, it is recommended to use the one
proposed in the non-normative Appendix C of \cite{MathML3}.

\subsection{Text and Math layout}

\subsubsection{Open Font Format}

Because math fonts generally contain very tall glyphs such as big integrals,
using typographic metrics is important to avoid excessive line spacing of text.
This behavior is specified in math fonts using the USE\_TYPO\_METRICS flag from
the OS/2 table \cite{OpenFontFormat3} and user agents must honor that flag.

Mathematical formulas can be viewed as an extension of standard text layout
and thus user agents must be able to perfom
complex text layout \cite{CTL} using fonts under the Open Font Format
\cite{OpenFontFormat3}. In particular, they must implement bidirectional
rendering and shaping of Arabic scripts.

Mathematical formulas may mix standard text with other graphical outlines
(e.g. fraction or top radical bars). For consistency, these outlines should
be rendered the same way as normal text. In particular, user agents must be
able to apply {\tt visibility} and {\tt color} CSS properties to them. They may
also support similar CSS properties for text such as {\tt text-shadow} or
{\tt opacity}.

Good mathematical rendering requires use of non-Unicode glyphs. Mathematical
fonts may only provide these glyphs when the {\tt math\lxAddClass{MATH}}
script tag is enabled and so user agents must ensure that the text within
MathML token elements is rendered with that script tag. Some characters like
primes already have script size by default and hence would be too small when
used in a script position.
Hence user agents must support glyph selections via the OpenType font feature
{\tt ssty\lxAddClass{MATH}} (Script Style) in order to display such
``prescripted'' characters with the appropriate size.
For bidirectional layout, Unicode defines character-level mirroring
to transform a character into its mirrored form,
for example U+0028 LEFT PARENTHESIS into U+0029 RIGHT PARENTHESIS.
User agents must also support the OpenType font feature
{\tt rtlm} (Right-to-left mirrored forms)
to allow glyph-level mirroring in cases where character-level is not enough
\cite{OpenFontFormat3}.
At the time of writing, Unicode does not distinguish between Chancery and
Spencerian style for the Unicode MATHEMATICAL SCRIPT characters. Some
mathematical fonts rely on {\tt salt} or {\tt ssXY} properties to provide
both styles. User agents may support the CSS {\tt font-variant-alternates}
property and corresponding OpenType font features to enable page authors
to get access to these styles \cite{CSS3Font} \cite{OpenFontFormat3}.

User agents must be able to read information from the
OpenType MATH table \cite{OpenFontFormat3}.
In particular they must be able to read the values from the
{\tt MathConstants\lxAddClass{MATH}} subtable.
They must also be able to use the {\tt MathVariants\lxAddClass{MATH}} subtable
to solve the following problem: given a particular default glyph shape and a
certain width or height, find a variant shape glyph (or a construct created by
putting several glyphs together) that has the required measurement.
More information are provided in see section \ref{Operators}.

\subsubsection{LaTeX}\label{LaTeX}

Mathematical rendering rules in this document are implicitly based on
MathML, OpenType MATH table and the TeXBook
\cite{MathML3} \cite{OpenFontFormat3} \cite{TeXBook}. In this section, we
describe more precisely some rules from the TeXBook.

In addition to concepts similar to MathML's {\tt displaystyle} and
{\tt scriptlevel}, LaTeX has
a ``cramped'' property, which is involved in the determination of script shifts.
It is initially unset on the {\tt math} element and in general
all the children inherits the ``cramped'' property from their parent.
However, it is set to true in the following children:

\begin{enumerate}
\item The denominator of an {\tt mfrac} element. See section \ref{Fractions}.
\item The subscripts of the {\tt msub}, {\tt msubsup},
  {\tt munder} and {\tt munderover} elements.
  See section \ref{ScriptAndLimitSchemata}.
\item The overscript of the
  {\tt mover} and {\tt munderover} elements when it is an {\tt accent} per
  the MathML specification. See section \ref{ScriptAndLimitSchemata}.
\item The children of the {\tt msqrt} and {\tt mroot}
  elements as well as the {\tt menclose} element that have {\tt radical}
  notation.
\end{enumerate}

To implement math spacing, the TeXBook defines eight basic types
({\tt Ord} for ordinary atoms, {\tt Op} for large operators,
{\tt Bin} for binary operations, {\tt Rel} for relations,
{\tt Open} for opening fences, {\tt Close} for closing fences,
{\tt Punct} for punctuations and {\tt Inner} for a delimited subformula) and
define an inter space for each pair of such types. In the present document,
we only follow the spacing algorithm of MathML 3: by default the inter space
is always zero and the spacing is produced by spacing elements like
{\tt mspace}, {\tt mphantom} or {\tt mpadded} or by the leading and trailing
space around embellished operators.

\subsection{CSS Styling}

\subsubsection{Properties}\label{CSSProperties}

User agents must support the CSS language \cite{CSS2} and must take special
styling into account when building the visual representation of the document.
Many of the MathML elements accept attributes with length value whose
general syntax is described in section 2.1.5.2 of the MathML specification
\cite{MathML3}. In general,
the syntax is compatible with \cite{CSS2} but user agents must handle
specificities of the MathML specification. In particular, the keywords
{\tt veryverythinmathspace},
{\tt verythinmathspace},
{\tt thinmathspace},
{\tt mediummathspace},
{\tt thickmathspace},
{\tt verythickmathspace},
{\tt veryverythickmathspace},
{\tt negativeveryverythinmathspace},
{\tt negativeverythinmathspace},
{\tt negativethinmathspace},
{\tt negativemediummathspace},
{\tt negativethickmathspace},
{\tt negativeverythickmathspace} and
{\tt negativeveryverythickmathspace} must be interpreted as their
equivalent {\tt em} value. Also, percent and unitless values must be interpreted
with respect to the appropriate reference value. Note that the {\tt mpadded}
element also accepts more general length values as discussed in section
\ref{mpadded}.

User agents must support at least the following properties:
\begin{enumerate}
\item {\tt display}: at least inline, block, inline-table, table-row,
  table-cell and none. It is used for the {\tt math} and tabular
  elements. Appropriate values may be specified in the user agent stylesheet
  as described in section \ref{UAStylesheet}.
\item {\tt direction}. The {\tt dir} attribute on the {\tt math}, {\tt mstyle},
  {\tt mrow} and {\tt token} elements must be mapped to that property.
\item {\tt font} property and its shorthands. The {\tt mathsize} attribute on
  the {\tt math}, {\tt mstyle} and token elements must mapped to that property.
\item {\tt background} and {\tt color}. The {\tt mathcolor} and
  {\tt mathbackground} attributes on presentation MathML elements must be
  mapped to these properties.
\item {\tt visibility}. It is is used for the {\tt mphantom} element and may be
  specified in the user agent stylesheet as described in section
  \ref{UAStylesheet}.
\end{enumerate}

User agents must support the {\tt displaystyle} attribute. This may be
implemented using a new {\tt mathml-math-style} property described in
table \ref{tab:CSSMathStyle}. The expected behavior may be completely
specified in the user agent stylesheet as described in section
\ref{UAStylesheet}.

\begin{table}
\begin{tabular}{ll}
  \emph{Name:} & {\tt 'mathml-math-style'} \\
  \emph{Value:} & display | inline \\
  \emph{Initial:} & inline \\
  \emph{Applies to:} & all elements \\
  \emph{Inherited:} & yes \\
  \emph{Percentages:} & N/A \\
  \emph{Media:} & visual \\
  \emph{Computed value:} & as specified \\
  \emph{Animatable:} & no \\
  \emph{Description:} & This value indicates the style to use for mathematical
  formulas. Formulas in {\tt display} mode will typically make more generous
  use of vertical space than inline formulas. Detailed rules are provided in
  section \ref{PresentationMarkup}.
\end{tabular}
\label{tab:CSSMathStyle}
\end{table}

User agents must support the {\tt mathvariant} attribute. This may be
implemented using a new {\tt mathml-math-variant} property described in
table \ref{tab:CSSMathVariant}. The expected behavior may be implemented by
mapping {\tt mathvariant} attributes on the {\tt math}, {\tt mstyle} and token
elements to {\tt mathml-math-variant}. Then during the rendering of text nodes,
the {\tt mathml-math-variant} value must be taken into account to remap
some characters to their equivalent code points, as specified by the MathML
specification. However, as indicated in section \ref{Identifier}, when
{\tt mathml-math-variant} is {\tt none} on an {\tt mi} element with a single
character, it must actually be treated as if the mathvariant was {\tt italic} ;
this can not be handled via CSS-only.

\begin{table}
\begin{tabular}{ll}
  \emph{Name:} & {\tt 'mathml-math-variant'} \\
  \emph{Value:} & {\tt "none" | "normal" | "bold" | "italic" | "bold-italic" |
  "double-struck" | "bold-fraktur" | "script" | "bold-script" | "fraktur" |
  "sans-serif" | "bold-sans-serif" | "sans-serif-italic" |
  "sans-serif-bold-italic" | "monospace" | "initial" | "tailed" | "looped" |
  "stretched"} \\
  \emph{Initial:} & none \\
  \emph{Applies to:} & All elements \\
  \emph{Inherited:} & yes \\
  \emph{Percentages:} & N/A \\
  \emph{Media:} & visual \\
  \emph{Computed value:} & as specified \\
  \emph{Animatable:} & no \\
  \emph{Description:} & When not {\tt none}, characters in text nodes
  should be rendered using the equivalent Unicode characters of specified
  mathvariant, taken from the Arabic Mathematical Alphabetic Symbols block
  (U+1EE00 to U+1EEFF), the Mathematical Alphanumeric Symbols block
  (U+1D400 to U+1D7FF), or the Letterlike Symbols block (U+2100 to U+214F)
  that represent "holes" in the alphabets in the
  Mathematical Alphanumeric Symbols block. If no such Unicode character
  exists for the specified mathvariant or if that property is {\tt none} then
  no transformations are performed and the original character is used for the
  rendering.
\end{tabular}
\label{tab:CSSMathVariant}
\end{table}

User agents must implement the {\tt scriptlevel} attributes and support
automatic incrementation of the {\tt scriptlevel} described in the MathML
specification. They may implement the {\tt scriptsizemultiplier} and
{\tt scriptminsize} attributes or instead just use the default values
(respectively {\tt 0.71} and {\tt 8pt}). These parameters affect the
{\tt font-size} property as follows: If the parent
{\tt font-size} is not already below the {\tt scriptminsize}, then the
child {\tt font-size} is computed by the following formula:
%
\begin{equation}
\text{FontSize}_{\text{Child}} =
{\max \left(
  \text{ScriptMinSize}_{\text{Parent}},
  {\text{FontSize}_{\text{Parent}} \times
    {\text{ScriptSizeMultiplier}_{\text{Parent}}^
      {\left(\text{FontSize}_{\text{Child}} -
            \text{FontSize}_{\text{Parent}}\right)}
  }}
\right)}\label{eq:fontsize}
\end{equation}

Since the {\tt font-size} change is typically done in a dedicated CSS module of
user agents, it may be convenient to introduce a new
{\tt mathml-script-level} property described in table \ref{tab:CSSScriptLevel}.
User agents may then implement the expected behavior by mapping the
{\tt scriptlevel} attribute on the {\tt math} and {\tt mstyle} elements to that
property and adding appropriate rules in the user agent stylesheet
(section \ref{UAStylesheet}). However as indicated in section
\ref{UnderscriptsOverscripts}, the {\tt scriptlevel} change for underscripts
and overscripts can not be handled via CSS-only.

If {\tt scriptsizemultiplier} and {\tt scriptminsize}
are supported, then they may also be implemented using new CSS properties
{\tt mathml-script-size-multiplier} and {\tt mathml-script-min-size}
described in tables \ref{tab:CSSScriptSizeMultiplier} and
\ref{tab:CSSScriptSizeMinSize} respectively. User agents may then implement
the expected behavior by mapping the {\tt scriptsizemultiplier}
and {\tt scriptminsize} attributes on the {\tt math} and {\tt mstyle} elements
to these properties.

\begin{table}
\begin{tabular}{ll}
  \emph{Name:} & {\tt 'mathml-script-level'} \\
  \emph{Value:} &
       {\tt \textless integer\textgreater |
         increment-by \textless integer\textgreater | auto} \\
  \emph{Initial:} & 0 \\
  \emph{Applies to:} & all elements \\
  \emph{Inherited:} & yes \\
  \emph{Percentages:} & N/A \\
  \emph{Media:} & visual \\
  \emph{Computed value:} &
  If the specified value is {\tt auto}, the computed value is obtained by
  incrementing the parent {\tt mathml-script-level} if and only if
  {\tt mathml-math-style} is {\tt inline}.
  If the specified value is {\tt increment-by \textless integer\textgreater},
  the computed value is obtained by
  incrementing the parent {\tt mathml-script-level} by the specified integer.
  Otherwise, the computed value is as specified. \\
  \emph{Animatable:} & no \\
  \emph{Description:} & This describes the scriptlevel of a mathematical
  formulas and affects the computation of {\tt font-size} as specified by
  equation \ref{eq:fontsize}.
\end{tabular}
\label{tab:CSSScriptLevel}
\end{table}

\begin{table}
\begin{tabular}{ll}
  \emph{Name:} & {\tt 'mathml-script-size-multiplier'} \\
  \emph{Value:} & \textless number\textgreater \\
  \emph{Initial:} & {\tt 0.71} \\
  \emph{Applies to:} & All elements \\
  \emph{Inherited:} & yes \\
  \emph{Percentages:} & N/A \\
  \emph{Media:} & visual \\
  \emph{Computed value:} & as specified \\
  \emph{Animatable:} & no \\
  \emph{Description:} & Specifies the multiplier to be used to adjust
       {\tt font-size} due to changes in scriptlevel as specified by equation
       \ref{eq:fontsize}.
\end{tabular}
\label{tab:CSSScriptSizeMultiplier}
\end{table}

\begin{table}
\begin{tabular}{ll}
  \emph{Name:} & {\tt 'mathml-script-min-size'} \\
  \emph{Value:} & \textless length\textgreater \\
  \emph{Initial:} & {\tt 8pt} \\
  \emph{Applies to:} & All elements \\
  \emph{Inherited:} & yes \\
  \emph{Percentages:} & N/A \\
  \emph{Media:} & visual \\
  \emph{Computed value:} & as specified \\
  \emph{Animatable:} & no \\
  \emph{Description:} & Specifies the minimum font size allowed due to changes
  in scriptlevel as specified by equation \ref{eq:fontsize}.
\end{tabular}
\label{tab:CSSScriptSizeMinSize}
\end{table}

\subsubsection{User Agent Stylesheet for MathML}\label{UAStylesheet}

Because mathematical formulas are generally written with special fonts, the
default user agent stylesheet must reset the CSS {\tt font-family} on the
{\tt math} element to {\tt serif}. User agents should then use their own
mechanism to try and interpret this {\tt serif} value on the {\tt math} element
as a font with an OpenType MATH table.

Below is an example on a stylesheet to style MathML elements on which the
User Agents may rely. Unfortunately, some rendering engines do not allow
universal selectors in their user agent stylesheets and so rules must be
expanded to list all possible MathML elements described in the present
document. For example {\tt mfrac > *} can be converted into
{\tt mfrac > mi, mfrac > mn, mfrac > mo, mfrac > mtext, mfrac > mspace,
mfrac > ms, mfrac > mrow, mfrac > mfrac, mfrac > msqrt, mfrac > mroot,
mfrac > mstyle, mfrac > merror, mfrac > mpadded, mfrac > mphantom,
mfrac > menclose, mfrac > msub, mfrac > msubsup, mfrac > munder, mfrac > mover,
mfrac > munderover, mfrac > mmultiscripts, mfrac > mtable, mfrac > maction}.

\begin{lstlisting}
@namespace url(http://www.w3.org/1998/Math/MathML);

/* The <math> element */
math {
  direction: ltr;
  display: inline;
  font-size: inherit;
  font-style: normal;
  font-family: serif;
  mathml-math-style: inline;
}
math[display="block"] {
  display: block;
  text-align: center;
  mathml-math-style: display;
}
math[display="inline"] {
  display: inline;
  mathml-math-style: inline;
}
math[displaystyle="false"] {
  mathml-math-style: inline;
}
math[displaystyle="true"] {
  mathml-math-style: display;
}

/* Links */
mrow[href] {
  color: blue;
}

/* Tabular elements */
mtable {
  display: inline-table;
  mathml-math-style: inline;
}
mtable[displaystyle="true"] {
  mathml-math-style: display;
}
mtable[frame="none"] {
  border: none;
}
mtable[frame="solid"] {
  border: solid thin;
}
mtable[frame="dashed"] {
  border: dashed thin;
}
mtr, mlabeledtr {
  display: table-row;
  vertical-align: baseline;
}
mlabeledtr > mtd:first-child {
  display: none;
}
mtd {
  display: table-cell;
  vertical-align: inherit;
  text-align: center;
  padding: 0.5ex 0.4em;
}

/* The <ms> element */
ms {
  display: inline;
}
ms:before, ms:after {
  content: "\0022"
}
ms[lquote]:before {
  content: attr(lquote)
}
ms[rquote]:after {
  content: attr(rquote)
}

/*  The <merror> element */
merror {
 outline: solid thin red;
 background-color: lightYellow;
}

/* The <mphantom> element */
mphantom {
  visibility: hidden;
}

/* Scriptlevel and displaystyle for other elements */
mstyle[displaystyle="false"] {
  mathml-math-style: inline;
}
mstyle[displaystyle="true"] {
  mathml-math-style: display;
}
mfrac > * {
  mathml-script-level: auto;
  mathml-math-style: inline;
}
mroot > :not(:first-child) {
  mathml-script-level: increment-by 2;
  mathml-math-style: inline;
}
msub > :not(:first-child),
msup > :not(:first-child),
msubsup > :not(:first-child),
mmultiscripts > :not(:first-child) {
  mathml-script-level: increment-by 1;
  mathml-math-style: inline;
}
munder > :not(:first-child),
mover > :not(:first-child),
munderover > :not(:first-child) {
  mathml-math-style: inline;
}
\end{lstlisting}
