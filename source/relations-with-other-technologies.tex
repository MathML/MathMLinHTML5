\section{Relation with other technologies}

\subsection{HTML5 Tree}

\subsubsection{DOM, HTML, SVG and Javascript}

User agents must use a HTML5 \cite{HTML5} parser to build a DOM tree
\cite{DOM1} from the source code of web pages. In particular, they must follow
the rules describing when elements must be in the MathML namespace
{\tt http://www.w3.org/1998/Math/MathML} and must
recognize the entity definitions from the HTML MathML entity set of
\cite{XMLEntities}. They must also take into account the following integration
points between SVG, MathML and HTML as allowed by \cite{ValidatorSchemas}:

\begin{enumerate}
\item The {\tt <math>} element can be used at any position permitted for
  phrasing content or inside SVG {\tt <foreignObject>} elements.
\item The {\tt <svg>} element can be used inside {\tt <annotation-xml>}
  elements with encoding {\tt SVG1.1} or {\tt image/svg+xml}.
\item The {\tt <html>} element and flow content can be used inside
  {\tt <annotation-xml>} elements with encoding {\tt application/xhtml+xml}
  or {\tt text/html}.
\item Any phrasing element can be used inside {\tt <mtext>} elements.
\end{enumerate}

From this DOM tree, user agents must provide a visual representation of the
document. The DOM tree may be dynamically modified using Javascript
\cite{ECMA262} and the user agents must keep the visual representation in
synchronization with the DOM tree.

When evaluating the SVG {\tt requiredExtensions} attribute \cite{SVG11},
user agents must claim support for the extension of name
{\tt http://www.w3.org/1998/Math/MathML}.
An algorithm to decide the visible child of the {\tt <semantics>} element is
proposed in section \ref{semantics}.

\subsubsection{MathML 3}

All MathML elements accepts the {\tt id}, {\tt class} and {\tt style} attributes
\cite{MathML3}.
They must be interpreted as described in section 3.2.5 of the
HTML5 specification \cite{HTML5} and in particular to specify a unique
identifier (to identify elements in links and scripting), affect CSS selectors
and {\tt getElementsByClassName()} and enable authors to do inline styling.

MathML 3 allows to use the {\tt href} attribute on any MathML element
\cite{MathML3}. In the present specification, it is only required to implement
{\tt href} on the {\tt mrow} element with the behavior described in 4.8
of the HTML5 specification for the {\tt a} element \cite{HTML5}. It is
recommended to make links visually distinguisable by default, for example by
adding a rule in the User Agent Stylesheet (section \ref{UAStylesheet}) such as
\begin{lstlisting}
mrow[href] {
  color: blue;
}
\end{lstlisting}

The toplevel {\tt math} element accepts the {\tt altimg}, {\tt altimg-width},
{\tt altimg-height}, {\tt altimg-valign} and {\tt alttext} attributes on it.
These attributes allow to specify fallback content and may just be ignored.
User agents that do not fully implement this specification may decide to
render this fallback content in some situation e.g. when they are specified or
under a preference option. In that case, an approximate implementation is to
render the {\tt math} element as a HTML {\tt img} element with the
value of {\tt altimg} as an {\tt src} attribute, the value of
{\tt alt} as a {\tt alttext} attribute, the value of {\tt altimg-width} as
the CSS {\tt width} property, the value of {\tt altimg-height} as
the CSS {\tt height} property and the value of {\tt altimg-valign} as
the CSS {\tt vertical-align} property \cite{HTML5} \cite{CSS2}.

The toplevel {\tt math} element also accepts the {\tt display} attribute,
{\tt mathcolor}, {\tt mathbackground} attributes as well as other attributes
from the {\tt mstyle} element. These attributes must be supported and
implementation can be achieved using specific rules in the User Agent Stylesheet
as described in section \ref{UAStylesheet}.

In general MathML elements or attributes that are not mentioned in this
specification may just be ignored. This includes deprecated attributes or
Content Markup descripted in chapter 4 of the MathML 3 specification
\cite{MathML3}.

\subsection{Text layout and Open Font Format}

User agents must be able to perfom complex text layout \cite{CTL} using
fonts under the Open Font Format \cite{OpenFontFormat3}. In particular, they
must implement bidirectional rendering and shaping of Arabic scripts.
User agents should be able to render some graphical outlines (e.g. fraction
or top radical bars) the same way as normal text and must be able to apply
{\tt visibility} and {\tt color} CSS properties to them. They may also
support similar CSS properties for text such as like {\tt text-shadow} or
{\tt opacity}.

User agents must render the text within MathML token elements with the
{\tt math} script tag \cite{OpenFontFormat3}. When determining the
text metrics, they must honor the USE\_TYPO\_METRICS flag from the OS/2 table
\cite{OpenFontFormat3}.
They must support glyph selections via the OpenType font
features {\tt ssty} (Script Style), {\tt flac}
(Flattened Accents over Capitals), {\tt dtls} (Dotless Forms)
and {\tt rtlm} (Right-to-left mirrored forms) \cite{OpenFontFormat3}.

User agents may also support the CSS {\tt font-variant-alternates} property
and corresponding OpenType font features \cite{CSS3Font} \cite{OpenFontFormat3}.
That way, font designers and page
authors can rely on it to provide a calligraphic style for the
Unicode Mathematical script characters. This allows to distinguish between
LaTeX's {\tt mathscr} and {\tt mathcal} commands.

User agents must be able to read information from the
OpenType MATH table \cite{OpenFontFormat3}.
In particular they must be able to read the values from the MathConstants
table. They must also be able to use the MathVariants subtable to solve the
following problem: given a particular default glyph shape and a
certain width or height, find a variant shape glyph (or a construct created by
putting several glyphs together) that has the required measurement.
[TODO: add reference to section about mo operators]

\subsection{CSS Styling}

\subsubsection{Properties}

User agents must support the CSS language \cite{CSS2} and take special styling
into account when building the visual representation of the document.
We assume that at least the following properties are supported:
%
\begin{enumerate}
\item {\tt display}: at least inline, block, inline-table, table-row,
  table-cell and none.
\item {\tt direction}
\item {\tt font} property and its shorthands.
\item {\tt background} and {\tt color}
\item {\tt visibility}
\end{enumerate}
%
In addition, this specification introduces new non-animatable CSS properties:
[TODO add reference / explanation]
%
\begin{enumerate}
\item {\tt mathml-script-level}
\item {\tt mathml-script-size-multiplier}
\item {\tt mathml-script-min-size}
\item {\tt mathml-math-variant}
\item {\tt mathml-math-style}
\end{enumerate}

The following MathML attributes must be mapped to CSS properties:
%
\begin{enumerate}
\item The {\tt mathcolor} and {\tt mathbackground} attributes on presentation
  MathML elements are mapped to {\tt color} and {\tt background} respectively.
\item The {\tt mathsize}, {\tt mathvariant} attributes on the {\tt math},
  {\tt mstyle} and token elements are mapped to {\tt font-size} and
  {\tt mathml-math-variant} respectively.
\item The {\tt dir} attribute on the {\tt math}, {\tt mstyle}, {\tt mrow} and
  {\tt token} elements are mapped to {\tt direction}.
\item The {\tt scriptlevel}, {\tt scriptminsize} and {\tt scriptsizemultiplier}
  attributes on the {\tt math} and {\tt mstyle} elements are mapped to
  {\tt mathml-script-level}, {\tt mathml-script-size-multiplier}
  and {\tt mathml-script-min-size} respectively.
\end{enumerate}

Many of the MathML elements accept attributes with length value whose
general syntax is described in section 2.1.5.2 of the MathML specification
\cite{MathML3}. In general,
the syntax is compatible with \cite{CSS2} but User Agents must handle
specificities of the MathML specification. In particular, the keywords
{\tt veryverythinmathspace},
{\tt verythinmathspace},
{\tt thinmathspace},
{\tt mediummathspace},
{\tt thickmathspace},
{\tt verythickmathspace},
{\tt veryverythickmathspace},
{\tt negativeveryverythinmathspace},
{\tt negativeverythinmathspace},
{\tt negativethinmathspace},
{\tt negativemediummathspace},
{\tt negativethickmathspace},
{\tt negativeverythickmathspace} and
{\tt negativeveryverythickmathspace} keywords must be interpreted as their
equivalent {\tt em} value. Also, percent and unitless values must be interpreted
with respect to the appropriate reference value. Note that the {\tt mpadded}
element also accepts more general length values as discussed in section
\ref{mpadded}.

\subsubsection{User Agent Stylesheet for MathML}\label{UAStylesheet}

Because mathematical formulas are generally written with special fonts, the
default user agent stylesheet must reset the CSS {\tt font-family} on the
{\tt math} element to {\tt serif}. User agents should then use their own
mechanism to try and interpret this {\tt serif} value on the {\tt math} element
as a font with an OpenType MATH table.

Below is an example on a stylesheet to style MathML elements on which the
User Agents may rely.
[TODO: Blink does not allow universal selectors in user agent
stylesheet. Should we use explicit tag names?]

\begin{lstlisting}
@namespace url(http://www.w3.org/1998/Math/MathML);

/* The <math> element */
math {
  direction: ltr;
  display: inline;
  font-size: inherit;
  font-style: normal;
  font-family: serif;
  mathml-math-style: inline;
}
math[display="block"] {
  display: block;
  text-align: center;
  mathml-math-style: display;
}
math[display="inline"] {
  display: inline;
  mathml-math-style: inline;
}
math[displaystyle="false"] {
  mathml-math-style: inline;
}
math[displaystyle="true"] {
  mathml-math-style: display;
}

/* Links */
mrow[href] {
  color: blue;
}

/* Tabular elements */
mtable {
  display: inline-table;
  border-collapse: separate;
  border-spacing: 0;
  mathml-math-style: inline;
}
mtable[displaystyle="true"] {
  mathml-math-style: display;
}
mtr, mlabeledtr {
  display: table-row;
  vertical-align: baseline;
}
mtd {
  display: table-cell;
  vertical-align: inherit;
  text-align: center;
}
mlabeledtr > mtd:first-child {
  display: none;
}

/* The <ms> element */
ms {
  display: inline;
}
ms:before, ms:after {
  content: "\0022"
}
ms[lquote]:before {
  content: attr(lquote)
}
ms[rquote]:after {
  content: attr(rquote)
}

/*  The <merror> element */
merror {
 outline: solid thin red;
 background-color: lightYellow;
}

/* The <mphantom> element */
mphantom {
  visibility: hidden;
}

/* Scriptlevel and displaystyle for other elements */
mstyle[displaystyle="false"] {
  mathml-math-style: inline;
}
mstyle[displaystyle="true"] {
  mathml-math-style: display;
}
mfrac > * {
  mathml-script-level: auto;
  mathml-math-style: inline;
}
mroot > :not(:first-child) {
  mathml-script-level: +2;
  mathml-math-style: inline;
}
msub > :not(:first-child),
msup > :not(:first-child),
msubsup > :not(:first-child),
mmultiscripts > :not(:first-child) {
  mathml-script-level: +1;
  mathml-math-style: inline;
}
munder > :not(:first-child),
mover > :not(:first-child),
munderover > :not(:first-child) {
  mathml-math-style: inline;
}
\end{lstlisting}
