\section{Relation with other technologies}

\subsection{DOM, HTML, SVG and Javascript}

User agents must use a HTML5 \cite{HTML5} parser to build a DOM tree
\cite{DOM1} from the source code of web pages. In particular, they must follow
the rules describing when elements must be in the MathML namespace
{\tt http://www.w3.org/1998/Math/MathML} and must
recognize the entity definitions from the HTML MathML entity set of
\cite{XMLEntities}. They must also take into account the following integration
points between SVG, MathML and HTML as allowed by \cite{ValidatorSchemas}:

\begin{enumerate}
\item The {\tt <math>} element can be used at any position permitted for
  phrasing content or inside SVG {\tt <foreignObject>} elements.
\item The {\tt <svg>} element can be used inside {\tt <annotation-xml>}
  elements with encoding {\tt SVG1.1} or {\tt image/svg+xml}.
\item The {\tt <html>} element and flow content can be used inside
  {\tt <annotation-xml>} elements with encoding {\tt application/xhtml+xml}
  or {\tt text/html}.
\item Any phrasing element can be used inside {\tt <mtext>} elements.
\end{enumerate}

From this DOM tree, user agents must provide a visual representation of the
document. The DOM tree may be dynamically modified using Javascript
\cite{ECMA262} and the user agents must keep the visual representation in
synchronization with the DOM tree.

When evaluating the SVG {\tt requiredExtensions} attribute \cite{SVG11},
user agents must claim support for the extension of name the MathML namespace
string.
An algorithm to decide the visible child of the {\tt <semantics>} element is
proposed in \ref{semantics}.

\subsection{Text layout and Open Font Format}

User agents must be able to perfom complex text layout \cite{CTL} using
fonts under the Open Font Format \cite{OpenFontFormat3}. In particular, they
must implement bidirectional rendering and shaping of Arabic scripts.
User agents should be able to render some graphical outlines (e.g. fraction
or top radical bars) the same way as normal text and thus should apply to them
similar CSS properties like {\tt text-shadow} or {\tt color}.

User agents must render the text within MathML token elements with the
{\tt math} script tag \cite{OpenFontFormat3}. When determining the
text metrics, they must honor the USE\_TYPO\_METRICS flag from the OS/2 table
\cite{OpenFontFormat3}.
They must support glyph selections via the OpenType font
features {\tt ssty} (Script Style), {\tt flac}
(Flattened Accents over Capitals), {\tt dtls} (Dotless Forms)
and {\tt rtlm} (Right-to-left mirrored forms) \cite{OpenFontFormat3}.

User agents may also support the CSS {\tt font-variant-alternates} property
and corresponding OpenType font features \cite{CSS3Font} \cite{OpenFontFormat3}.
That way,  font designers and page
authors can rely on it to provide a calligraphic style for the
Unicode Mathematical script characters. This allows to distinguish between
LaTeX's {\tt mathscr} and {\tt mathcal} commands.

User agents must be able to read information from the
OpenType MATH table \cite{OpenFontFormat3}.
In particular they must be able to read the values from the MathConstants
table. They must also be able to use the MathVariants subtable to solve the
following problem: given a particular default glyph shape and a
certain width or height, find a variant shape glyph (or a construct created by
putting several glyphs together) that has the required measurement.
See ???? (section about mo operators) for more details.

\subsection{CSS}

\subsubsection{Properties}

User agents must support the CSS language \cite{CSS2} and take special styling
into account when building the visual representation of the document. We assume
that at least the following properties are supported:

\begin{enumerate}
\item {\tt display}: at least inline, block, inline-table, table-row,
  table-cell and none.
\item {\tt direction}
\item {\tt font} property and its shorthands.
\item {\tt background} and {\tt color}
\item {\tt visibility}
\end{enumerate}

In addition, this specification introduces new non-animatable CSS properties:

\begin{enumerate}
\item {\tt mathml-script-level}
\item {\tt mathml-script-size-multiplier}
\item {\tt mathml-script-min-size}
\item {\tt mathml-math-variant}
\item {\tt mathml-math-style}
\end{enumerate}

\subsubsection{Attributes mapped to CSS properties}\label{mappedAttributes}

The following MathML attributes are mapped to CSS properties:

\begin{enumerate}
\item The {\tt mathcolor} and {\tt mathbackground} attributes on presentation
  MathML elements are mapped to {\tt color} and {\tt background} respectively.
\item The {\tt mathsize}, {\tt mathvariant} attributes on the {\tt math},
  {\tt mstyle} and token elements are mapped to {\tt font-size} and
  {\tt mathml-math-variant} respectively.
\item The {\tt dir} attribute on the {\tt math}, {\tt mstyle}, {\tt mrow} and
  {\tt token} elements are mapped to {\tt direction}.
\item The {\tt scriptlevel}, {\tt scriptminsize} and {\tt scriptsizemultiplier}
  attributes on the {\tt math} and {\tt mstyle} elements are mapped to
  {\tt mathml-script-level}, {\tt mathml-script-size-multiplier}
  and {\tt mathml-script-min-size} respectively.
\end{enumerate}

\subsubsection{Box Model}

User agents must also follow the rules described in section 4.7.14
Embedded content, MathML of \cite{HTML5}, in particular the equivalence with
implicit {\tt mtext} and {\tt merror} to handle non-conforming MathML markup.

MathML elements have the following box model. The {\tt <math>} root may have
inline or block display while tabular MathML elements have table, table-row and
table-cell display (see ????). The {\tt <math>} and {\tt <mtd>} elements
generate an anonymous {\tt <mrow>} MathML box child that contains the boxes of
their children and use the box parameters below to layout this anonymous
{\tt <mrow>} box.
No line breaking can happen within MathML boxes and the min-content width is
equal to the max-content width, these are simply called the intrinsic width.
Each MathML box has the following parameters:

\begin{enumerate}
\item intrinsic width $W$.
\item logical width $w$, ascent above the origin $a$ and descent below the
  origin $b$. The height is then $a+b$.
\item ink ascent $A$ and descent $B$, corresponding to the exact box enclosing
  ink of text and bars within the MathML box.
\end{enumerate}

\begin{figure}
\centering
\begin{tikzpicture}[yscale=-1]
  \draw[->] (0,0) -- (15,0) node[below] {$x$};
  \draw[->] (0,0) -- (0,6) node[right] {$y$};
  \draw[dashed,blue] (0,-2) -- (13.5,-2) -- (13.5,4) -- (0,4) -- cycle;
  \MathMLBox{0}{0}{1}{1}{red}
  \MathMLBoxMetrics{0}{0}{1}{1}{red}{1}
  \MathMLBox{6}{2}{1.5}{1}{green}
  \MathMLBoxMetrics{6}{2}{1.5}{1}{green}{2}
\end{tikzpicture}
\label{MathMLBoxModel}
\end{figure}

For MathML element containing only text nodes or foreign elements, we assume
that the content is simple enough to determine these values. For example,
in most case, this is just a single text node and $w = W$.
A MathML element containing only other MathML elements follow special rules to
layout their children at position $(x_i,y_i)$ with parameters
$W_i,w_i,a_i,b_i,A_i,B_i$. Then as a general rule, its box parameters are given
by taking the union of child boxes, that is:
%
$$w = \left(\max_{1 \leq i \leq n } {x_i + w_i}\right) -
      \left(\min_{1 \leq i \leq n} x_i\right)$$
$$W = \left(\max_{1 \leq i \leq n } {x_i + W_i}\right) -
      \left(\min_{1 \leq i \leq n} x_i\right)$$
$$a = \max_{1 \leq i \leq n } {a_i - y_i}$$
$$b = \max_{1 \leq i \leq n } {b_i + y_i}$$
$$A = \max_{1 \leq i \leq n } {A_i - y_i}$$
$$B = \max_{1 \leq i \leq n } {B_i + y_i}$$
%

Note that the schemas in this specification are drawn assuming left-to-right
directionality. If the CSS {\tt direction} is set to right-to-left, then the
elements should be layout by making the $x$-axis pointing from right-to-left.

Detailed rules as well as possible additional spacing are given in sections ???.

\subsubsection{User Agent Stylesheet for MathML}

Because mathematical formulas are generally written with special fonts, the
default user agent stylesheet must reset the CSS {\tt font-family} on the
{\tt math} element to {\tt serif}. User agents should then use their own
mechanism to try and interpret this {\tt serif} value on the {\tt math} element
as a font with an OpenType MATH table.

User agents may use the following user agent stylesheet to style MathML
elements. (TODO: Blink does not allow universal selectors in user agent
stylesheet. Should we use explicit tag names?)

\begin{lstlisting}
@namespace url(http://www.w3.org/1998/Math/MathML);

/* The <math> element */
math {
  direction: ltr;
  display: inline;
  font-size: inherit;
  font-style: normal;
  font-family: serif;
  mathml-math-style: inline;
}
math[display="block"] {
  display: block;
  text-align: center;
  mathml-math-style: display;
}
math[display="inline"] {
  display: inline;
  mathml-math-style: inline;
}
math[displaystyle="false"] {
  mathml-math-style: inline;
}
math[displaystyle="true"] {
  mathml-math-style: display;
}

/* Tabular elements */
mtable {
  display: inline-table;
  border-collapse: separate;
  border-spacing: 0;
  mathml-math-style: inline;
}
mtable[displaystyle="true"] {
  mathml-math-style: display;
}
mtr, mlabeledtr {
  display: table-row;
  vertical-align: baseline;
}
mtd {
  display: table-cell;
  vertical-align: inherit;
  text-align: center;
}
mlabeledtr > mtd:first-child {
  display: none;
}

/* The <ms> element */
ms {
  display: inline;
}
ms:before, ms:after {
  content: "\0022"
}
ms[lquote]:before {
  content: attr(lquote)
}
ms[rquote]:after {
  content: attr(rquote)
}

/*  The <merror> element */
merror {
 outline: solid thin red;
 background-color: lightYellow;
}

/* The <mphantom> element */
mphantom {
  visibility: hidden;
}

/* Scriptlevel and displaystyle for other elements */
mstyle[displaystyle="false"] {
  mathml-math-style: inline;
}
mstyle[displaystyle="true"] {
  mathml-math-style: display;
}
mfrac > * {
  mathml-script-level: auto;
  mathml-math-style: inline;
}
mroot > :not(:first-child) {
  mathml-script-level: +2;
  mathml-math-style: inline;
}
msub > :not(:first-child),
msup > :not(:first-child),
msubsup > :not(:first-child),
mmultiscripts > :not(:first-child) {
  mathml-script-level: +1;
  mathml-math-style: inline;
}
munder > :not(:first-child),
mover > :not(:first-child),
munderover > :not(:first-child) {
  mathml-math-style: inline;
}
\end{lstlisting}
